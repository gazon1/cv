% Created 2020-08-20 Чт 10:19
% Intended LaTeX compiler: pdflatex
\documentclass[11pt]{article}
\usepackage[utf8]{inputenc}
\usepackage[T1]{fontenc}
\usepackage{graphicx}
\usepackage{grffile}
\usepackage{longtable}
\usepackage{wrapfig}
\usepackage{rotating}
\usepackage[normalem]{ulem}
\usepackage{amsmath}
\usepackage{textcomp}
\usepackage{amssymb}
\usepackage{capt-of}
\usepackage{hyperref}
\usepackage[profilePic={photo},profilePicWidth=60pt]{myCV}
\usepackage[T2A]{fontenc}
\date{}
\title{Дробин Максим}
\hypersetup{
 pdfauthor={max},
 pdftitle={Дробин Максим},
 pdfkeywords={},
 pdfsubject={},
 pdfcreator={Emacs 26.3 (Org mode 9.1.9)}, 
 pdflang={English}}
\begin{document}

\maketitle
\section{Контакты}
\label{sec:org25c2c82}
\subsection{Почта}
\label{sec:orgaffa8a4}
\href{mailto:drobin.me@yandex.ru}{drobin.me@yandex.ru}

\subsection{тел(+ телеграм/whatapp): +7-977-277-42-61}
\label{sec:org64386a6}
\subsection{fb: \url{https://www.facebook.com/profile.php?id=100042101110525}}
\label{sec:org4a47960}
\subsection{Linkedin: \url{https://www.linkedin.com/in/maxim-drobin-a11b05154/}}
\label{sec:org5e46e2b}
\subsection{Другое}
\label{sec:orga355af7}
\href{https://www.kaggle.com/malahai}{Kaggle}
\href{https://github.com/gazon1/}{Github}
\href{https://boosters.pro/user/Malahai}{Boosters}
\section{Опыт работы}
\label{sec:org8594642}
\subsection{Июнь 2018 - ноябрь 2018. Инженер-программист в Модульбанке}
\label{sec:orgbd03b36}
Автоматизировал бизнес процессы в маркетинге: 
\begin{itemize}
\item Автоматическое обновление лидов в БД по расписанию
\item Автоматический запуск почтовых рекламных кампаний на новых лидах по тригеру
\item Передача новых лидов на обзвон через скрипт
\end{itemize}

\texttt{Стект технологий:}
Python3, SQL, emarsys API, mytarget API, google marketing API
\subsection{Февраль 2019 - \ldots{}}
\label{sec:org9d4dfa9}
Ведущий психологических курсов по soft skills в МФТИ
\begin{itemize}
\item Ведущий курса antisoft skills в МФТИ по развитию личностных навыков под супервизией нарративных практиков
\item Организовал и провёл курс по нарративной практике для студентов волонтеров из проекта "Мне не все равно"
\end{itemize}
\subsection{Июль 2019 - сентябрь 2019. Junior data scientist в Тинькофф банке}
\label{sec:org15cb8f7}
\begin{itemize}
\item исправил геоматчинг и названия торговых точек на официальные, понятные, на русском языке в транзакциях. До этого названия магазина было обозначено через транслитерацию. Иногда это было не название торговой точки, а название юридического лица. Еще, координатой многих магазинов в Москве был обозначен центр Москвы. Исправил их на координаты самих магазинов.
\end{itemize}

\texttt{Cтек технологий:} python, NER, SQL, hive, greenplum, zepellin, bash, python, keras, linux
\subsection{Октябрь 2019 - \ldots{}}
\label{sec:org5369fc9}
"\href{https://docs.google.com/presentation/d/1AiwyzLKDgDXIaclUwoyGgT-VDJMtJYJQqAMVfbNlB40/edit\#slide=id.p1}{HighCar}"
Backend developer \& PR manager
Работа над своим проектом - каршеринг электромобилей в Ереване.
\begin{itemize}
\item В апреле 2020 участвовали в преакселераторе МФТИ "инженер 4.0", где провел анализ рынка и конкурентов
\item Разрабатываю backened часть на django
\item Подготовил питч и выступил на нескольких elevator-pitch перед инвесторами
\end{itemize}
\section{Образование}
\label{sec:org92af181}
\subsection{МФТИ (ГУ) (2015 – 2019)}
\label{sec:orgee00e1f}
ПМФ, ФРТК, кафедра - ЦИТИС

Бакалавр, тема ВКР

\begin{center}
\textbf{\emph{Прогнозирование температуры во времени и пространстве}}
\end{center}
\subsection{МФТИ (ГУ) (2019 - 2021)}
\label{sec:org803de67}
ПМФ, ФРТК, кафедра - ЦИТИС

Магистр, тема ВКР
\begin{center}
\textbf{\emph{Применение ансамблей алгоритмов в Рекомендательной системе}}
\end{center}

\section{Навыки}
\label{sec:orgd9d1b0e}
\subsection{Иниструменты и технологии}
\label{sec:org4cc52f8}
Python, SQL, Linux, Emacs, Git, bash
\subsection{Коммуникабельность}
\label{sec:org214a364}
\begin{itemize}
\item Провёл сентябрь 2019 - апрель 2020 курс по soft skills в МФТИ, учился год 2019-2020 на психолога, поэтому смогу общаться с клиентом на его
\end{itemize}
языке и укреплять командный дух в компании и своей команде. Давал примеры заданий, работал в парах и тройках, создавал атмосферу на занятии
\subsection{Другие}
\label{sec:org9f8f9a6}
\begin{itemize}
\item Pytorch, NumPy, pandas, xgboost, sklearn, fasttext, pickle, scipy, nltk, regex, tqdm, django, pycharm, requests, deep learning, unit testing
\end{itemize}
\section{It проекты, в которых участвовал}
\label{sec:orgc5efe9f}
\subsection{Выпускная квалификационная работа в бакалавриате "Прогнозирование температуры во времени и пространства"}
\label{sec:org8c5795e}
\url{https://github.com/gazon1/diplom-bachelor}. Автоматизировал сбор данных с метеорологических станций по API через python3 и библиотеку requests. Обучил нейросеть на keras и ансамбль деревье по методу градиентного бустинга и сравнил качество прогноза для городов вблизи Лондона.

\subsection{Прототип навигации в колекции видео Постнауки методом тематического моделирования}
\label{sec:orge025a84}
\url{https://github.com/gazon1/post-nauka-project/blob/master/PostnaukaPeerReview.ipynb}

\subsection{Классификатор по предсказанию категории обьявления по цене, описанию, заголовку}
\label{sec:orgcc9c9d3}
Итоговое решение - стекинг FastText классификатора и xgboost
\url{https://github.com/gazon1/testing-task-Avito}

\subsection{HighCar - каршеринг электромобилей в Ереване.}
\label{sec:org08b329d}
Презентация для elevator-pitch \url{https://docs.google.com/presentation/d/1AiwyzLKDgDXIaclUwoyGgT-VDJMtJYJQqAMVfbNlB40/edit?usp=sharing}

\subsection{Sentiment analysis of movie review. Django}
\label{sec:orgcb2cfb6}
Приложение на django. Оно состоит из одной формы - в него можно написать
отзыв на фильм на англ и посмотреть тональность отзыва. Тональность предсказывается
Bert модель.
\url{https://github.com/gazon1/GreenAtom}

\section{Kaggle Deep Learning/ML Competitions}
\label{sec:org954cdd8}
\subsection{Top 52\% - \href{https://cups.mail.ru/results/41?period=past\&round\_id=430}{Telecom Data Cup (all cups mail ru)}}
\label{sec:org9e64077}
\subsection{Топ 25\% - \href{https://boosters.pro/championship/modulbank1}{Modulbank AI Hack MSK (boosters)}}
\label{sec:org48a3952}
\subsection{Топ 6\% - \href{https://boosters.pro/championship/rekko\_challenge/overview}{REKKO CHALLENGE (boosters)}}
\label{sec:org52fc3c4}
\url{https://github.com/gazon1/Recco-challenge}

\subsection{Топ 13\% - \href{https://www.kaggle.com/c/competitive-data-science-predict-future-sales}{Predict Future Sales (kaggle)}}
\label{sec:orgbb3a610}
\url{https://github.com/gazon1/1c}

\subsection{Топ 68\% - \href{https://www.kaggle.com/c/2019s-neuralnet-track}{Нейронные сети. ДЗ 1.}}
\label{sec:org3052c2b}
\subsection{Топ 40\% - \href{https://www.kaggle.com/c/nn-track-2019-spring-hw2}{Нейронные сети. ДЗ 2.}}
\label{sec:org7f82827}
\subsection{Топ 59\%- \href{https://www.kaggle.com/c/neuralnetworkshomework4/leaderboard}{Neural Networks Homework 4 (ResNet)}}
\label{sec:org4a14793}
\subsection{Топ 80\% - \href{https://www.kaggle.com/c/ieee-fraud-detection}{IEEE-CIS Fraud Detection}}
\label{sec:org121bbdd}
\subsection{Топ 81\% - \href{https://www.kaggle.com/c/champs-scalar-coupling}{Predicting Molecular Properties}}
\label{sec:org7126578}
\section{Другие проекты}
\label{sec:orgf0c4a6e}
\subsection{Курс для волонтеров}
\label{sec:org3d713ae}
Организовал и провел курс по нарративной практике для студентов-волонтеров проекта "Мне не все равно"
\href{https://www.youtube.com/watch?v=EDkDUp0PgPE\&list=PL7GczH8KmOkD5QFvkeFVhJj6aGqpHkmeL\&index=12\&t=0s}{Мое выступление на край-фесте 6-7 июня 2020г по этому курсу}
\subsection{Зимний лагерь в Бельско-Устьенском детском доме интернате для детей с особенностями развития}
\label{sec:orgb99c82d}
\begin{itemize}
\item Помог провести за смену около 14 занятий с детьми в Бельско-Устьенском детском доме интернате для детей с особенностями развития. Всего смена длилась около недели
\item Проводил чайные церемонии для волонтеров
\end{itemize}
Emacs 26.3 (Org mode 9.1.9)
\end{document}
