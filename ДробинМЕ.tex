% Created 2020-05-28 Чт 23:48
% Intended LaTeX compiler: pdflatex
\documentclass[11pt]{article}
\usepackage[utf8]{inputenc}
\usepackage[T1]{fontenc}
\usepackage{graphicx}
\usepackage{grffile}
\usepackage{longtable}
\usepackage{wrapfig}
\usepackage{rotating}
\usepackage[normalem]{ulem}
\usepackage{amsmath}
\usepackage{textcomp}
\usepackage{amssymb}
\usepackage{capt-of}
\usepackage{natbib}
\usepackage[linktocpage,pdfstartview=FitH,colorlinks,
linkcolor=blue,anchorcolor=blue,
citecolor=blue,filecolor=blue,menucolor=blue,urlcolor=blue]{hyperref}
\usepackage[profilePic={photo},profilePicWidth=60pt]{myCV}
\usepackage[T2A]{fontenc}
\date{}
\title{Дробин Максим}
\hypersetup{
 pdfauthor={max},
 pdftitle={Дробин Максим},
 pdfkeywords={},
 pdfsubject={},
 pdfcreator={Emacs 26.3 (Org mode 9.1.9)}, 
 pdflang={English}}
\begin{document}

\maketitle
\section{Контакты}
\label{sec:orgb140f8b}
\subsection{Почта}
\label{sec:org932ad7a}
\href{mailto:drobin.me@yandex.ru}{drobin.me@yandex.ru}

\subsection{тел(+ телеграм/whatapp): +79772774261}
\label{sec:orgdff1f87}
\section{Языки}
\label{sec:orgf4d4dbb}
Английский - Upper-intermediate
\section{Опыт работы}
\label{sec:orga6f9032}
\subsection{Июнь 2018 - ноябрь 2018. Инженер-программист в Модульбанке}
\label{sec:org0c71276}
\begin{itemize}
\item Строил модели для расчета LTV
\item С помощью Python автоматически выкачивал данные из внешней БД(mixpanel) в локальную БД
\item Участвовал в проектировании схем в БД
\item Занимался автоматизацией бизнес процессов в маркетинге: сделал скрипт, который автоматически обновлял информацию в маркетинговых кабинетах(AdWords, яндекс аудитории, My Target) из БД
\item Занимался автоматизацией маркетинговой кампании по тригерам почтовой рассылки(Emarsys): письмо дошло/не дошло, адресант его открыл или нет, прошел ли по рекламной ссылке
\item Участвовал в обсуждение бизнес процессов
\end{itemize}
\subsection{Июль 2019 - сентябрь 2019. ml-инженер в Тинькофф банке}
\label{sec:orgee591b2}
\begin{itemize}
\item исправил геоматчинг и названия торговых точек на официальные, понятные, на русском языке в транзакциях. До этого названия магазина было обозначено через транслитерацию. Иногда это было не название торговой точки, а название юридического лица. Еще, координатой многих магазинов в Москве был обозначен центр Москвы. Исправил их на координаты самих магазинов.
\end{itemize}
\subsection{Сентябрь 2019 - \ldots{}}
\label{sec:orgdbb1fc6}
\begin{itemize}
\item Ведущий курса antisoft skills в МФТИ по развитию личностных навыков под супервизией Марии Тиуновой - нарративного практика.
\item Работа над своим стартапом "\href{https://docs.google.com/presentation/d/1AiwyzLKDgDXIaclUwoyGgT-VDJMtJYJQqAMVfbNlB40/edit\#slide=id.p1}{HighCar}" - каршеринг электромобилей в Ереване. В апреле 2020 участвовали в преакселераторе МФТИ "инженер
\end{itemize}
4.0". Мой роль - django backened developer и подготовка питчей и выступление.
\section{Образование}
\label{sec:org24867c9}
\subsection{МФТИ (ГУ) (2015 – 2019)}
\label{sec:orgb3ac79a}
ПМФ, ФРТК, кафедра - ЦИТИС

Бакалавр, тема ВКР

\begin{center}
\textbf{\emph{Прогнозирование температуры во времени и пространстве}}
\end{center}
\subsection{МФТИ (ГУ) (2019 - 2021)}
\label{sec:org8f3a2f6}
ПМФ, ФРТК, кафедра - ЦИТИС

Магистр, тема ВКР
\begin{center}
\textbf{\emph{Применение ансамблей алгоритмов в Рекомендательной системе}}
\end{center}

\section{Навыки}
\label{sec:org9d57f95}
\subsection{Отраслевые}
\label{sec:org1aeb4ea}
Машинное обучение, программирование, алгоритмы, анализ данных, математика, статистика
\subsection{Иниструменты и технологии}
\label{sec:orge3bf59a}
Python, SQL, Linux, Emacs, Git
\subsection{Коммуникабельность}
\label{sec:org6ebf45f}
\begin{itemize}
\item Был ведущим психологического курса "Антидеконструкторское бюро: навык, которые определяешь ты сам" на физтехе.
\end{itemize}
Давал примеры заданий, работал в парах и тройках, создавал атмосферу на занятии
\begin{itemize}
\item Участвовал в паре хакатонов, делал совместные учебные проекты, участвовал в соревнание на kaggle в команде
\end{itemize}
\subsection{Другие}
\label{sec:orgfe6cae8}
Pytorch, NumPy, pandas, xgboost, sklearn, fasttext, pickle, scipy, nltk, regex, tqdm, django, pycharm, requests, deep learning, unit testing

\section{EXTRA-CURRICULAR ACTIVITIES}
\label{sec:orgc8bc05a}
\subsection{Март 2018 – Май 2018}
\label{sec:orgd3b24ba}
Хакатон/соревнование от Модульбанка на booster.pro, Хакатон проходил в 2 этапа. 1й - на сайте booster.pro,
шел 2 месяца - нужно было построить модель классификации по анонимизированным данным(попал в топ 25\%). 2й - более творческий 
этап. Он проходил 2 дня. По дополнительным данным нужно было предложить свое творческое решение, которое может принести бизнессу
прибыль. Мы построили ансамбль из алгоритмов, которые в некотором приближении давал систему рекомендации. Заняли номинацию "За лучший код"!

\subsection{Декабрь 2017}
\label{sec:orgd626151}
Хакатон в МФТИ от global changers, командный хакатон. Задача была от Сибура - найти во временном ряду аномалии. 
Аномалии размечены не были. Заняли 2е место
\subsection{Апрель 2018 - май 2018}
\label{sec:org20d0c77}
\href{https://boosters.pro/championship/modulbank1}{Modulbank AI Hack MSK (boosters)} - топ 25\%
\subsection{Декабрь 2018 - Январь 2019}
\label{sec:orgb32b996}
Telecom Data Cup (ml bootcamp) - топ 50\%
\subsection{Февраль 2019 - апрель 2019}
\label{sec:orgb6a52ae}
okko competition по рекомендательным системам на \href{http://\\boosters.pro}{boosters} - топ6
\subsection{Июнь 2019}
\label{sec:org9b5a687}
Tele2 hack - 3е место
\section{Курсы}
\label{sec:orgfe4791a}
\subsection{Февраль 2017 - июнь 2017}
\label{sec:orgfcb725d}
\begin{itemize}
\item Курс по Java от NetCracker
\end{itemize}
\subsection{Сентябрь 2017 – Май 2018}
\label{sec:orga5d05fc}
\begin{itemize}
\item Парадигмы бизнесс программирования - курс от Никс по SQL
\end{itemize}
\subsection{September 2017 – Декабрь 2017}
\label{sec:org5eefe67}
\begin{itemize}
\item DMIA - курс по машинному обучению от яндекса
\end{itemize}
\subsection{Инюнь 2017 – Ноябрь 2017}
\label{sec:org7075b0d}
\begin{itemize}
\item \href{https://www.coursera.org/account/accomplishments/records/W46YJPAQ368V?utm\_source=link\&utm\_medium=certificate\&utm\_content=cert\_image\&utm\_campaign=sharing\_cta\&utm\_product=course}{Математика и Python для анализа данных}
\item \href{https://www.coursera.org/account/accomplishments/records/76L9YZ5TLEGL?utm\_source=link\&utm\_medium=certificate\&utm\_content=cert\_image\&utm\_campaign=sharing\_cta\&utm\_product=course}{Обучение на размеченных данных}
\item \href{https://www.coursera.org/account/accomplishments/records/2ECJXE69PBZH?utm\_source=link\&utm\_medium=certificate\&utm\_content=cert\_image\&utm\_campaign=sharing\_cta\&utm\_product=course}{Поиск структуры в данных}
\end{itemize}
\subsection{Февраль 2019 - Май 2019}
\label{sec:orgeaebe6a}
\begin{itemize}
\item курс от mail ru по deep learning
\item первые 3 курса по deep learning от Andrew NG
\end{itemize}
\subsection{Сентябрь 2019 - май 2020}
\label{sec:orgaf0dab0}
Курсы по тех. предпринимательству на основе своего стартапа "\href{https://docs.google.com/presentation/d/1AiwyzLKDgDXIaclUwoyGgT-VDJMtJYJQqAMVfbNlB40/edit?usp=sharing}{HighCar}"
\begin{itemize}
\item \href{https://www.coursera.org/account/accomplishments/records/UF4QZY62S5XW?utm\_source=link\&utm\_medium=certificate\&utm\_content=cert\_image\&utm\_campaign=sharing\_cta\&utm\_product=course}{Искусство системного инжиниринга и менеджмента 2.0}
\item \href{https://www.coursera.org/account/accomplishments/records/5HJ67CLZ74A4?utm\_source=link\&utm\_medium=certificate\&utm\_content=cert\_image\&utm\_campaign=sharing\_cta\&utm\_product=course}{Введение в системное проектирование}
\item \href{https://www.coursera.org/account/accomplishments/records/UTWB2YAP5Q6M?utm\_source=link\&utm\_medium=certificate\&utm\_content=cert\_image\&utm\_campaign=sharing\_cta\&utm\_product=course}{Методы и инструменты системного проектирования}
\item \href{https://www.coursera.org/account/accomplishments/records/VC79PW276TKT?utm\_source=link\&utm\_medium=certificate\&utm\_content=cert\_image\&utm\_campaign=sharing\_cta\&utm\_product=course}{Бизнес-процессы, организационное проектирование, механизмы и системы управления}
\end{itemize}
\subsection{Другое}
\label{sec:org4f3af88}
\begin{itemize}
\item \href{https://stepik.org/cert/347561}{Нейронные сети и компьютерное зрение от Samsung AI centre}
\item ods
\end{itemize}
\section{Ссылки}
\label{sec:orgb1e4b7b}
\subsection{hackerrank}
\label{sec:org30ba6ce}
\url{https://www.hackerrank.com/newbox2517}
\subsection{github}
\label{sec:orgd517014}
\href{https://github.com/gazon1/}{gazon1}
\subsection{boosters}
\label{sec:org62636d1}
\href{https://boosters.pro/user/Malahai}{Malahai}
\subsection{kaggle}
\label{sec:org7bee6a7}
\href{https://www.kaggle.com/malahai}{malahai}
Emacs 26.3 (Org mode 9.1.9)
\end{document}
